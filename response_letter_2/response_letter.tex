
%%%%%%%%%%%%%%%%%%%%%%%%%%%%%%%%%%%%%%%%%
% Professional Formal Letter
% LaTeX Template
% Version 1.0 (28/12/13)
%
% This template has been downloaded from:
% http://www.LaTeXTemplates.com
%
% Original author:
% Brian Moses (http://www.ms.uky.edu/~math/Resources/Templates/LaTeX/)
% with extensive modifications by Vel (vel@latextemplates.com)
%
% License:
% CC BY-NC-SA 3.0 (http://creativecommons.org/licenses/by-nc-sa/3.0/)
%
%%%%%%%%%%%%%%%%%%%%%%%%%%%%%%%%%%%%%%%%%

% may not be compatible with rest of template
% \documentclass[journal=jpcbfk,manuscript=article]{achemso}

%----------------------------------------------------------------------------------------
%	PACKAGES AND OTHER DOCUMENT CONFIGURATIONS
%----------------------------------------------------------------------------------------

%\documentclass[11pt,a4paper,draft]{letter} % Specify the font size (10pt, 11pt and 12pt) and paper size (letterpaper, a4paper, etc) 
\documentclass[11pt,a4paper]{letter} % Specify the font size (10pt, 11pt and 12pt) and paper size (letterpaper, a4paper, etc) 
\usepackage{letterbib}
\usepackage{graphicx} % Required for including pictures
\usepackage{microtype} % Improves typography
\usepackage{fancyhdr}
\usepackage[T1]{fontenc} % Required for accented characters
\usepackage[sort&compress,numbers,super]{natbib}
\usepackage{caption}
\usepackage{newfloat}
\usepackage{float}
\usepackage{url}
\usepackage{color}
\usepackage{soul}
\usepackage[svgnames]{xcolor}
\usepackage{framed}
\usepackage{calligra}
\usepackage{fp}
\FPset\marginWidth{0.7}	% margin width (all sides) in inches
\usepackage[margin=\marginWidth in]{geometry}
\usepackage{marginfix}
\usepackage[colorlinks=true,
            linkcolor=blue,
            urlcolor=blue,
            citecolor=blue]{hyperref}
\usepackage[font={small,sf}, singlelinecheck=false]{caption}
\usepackage{capt-of}           

% Create a new command for the horizontal rule in the document which allows thickness specification
\makeatletter

\def\vhrulefill#1{\leavevmode\leaders\hrule\@height#1\hfill \kern\z@}
\makeatother

%more new commands and definitions
%\newcommand*{\rood}[1]{\color{red}{#1}}
\newcommand*{\rood}[1]{{\color{red}{#1}}}
\newcommand*{\roodr}[1]{{\color{red}{#1}}}
\newcommand*{\noteg}[1]{\textcolor{green}{[[#1]]}}		% notes 
\newcommand*{\noter}[1]{{\color{red}{[[#1]]}}}

\newcommand*{\blauw}[1]{\color{blue}{#1}}
\newcommand*{\groen}[1]{\color{green}{#1}}
\newcommand{\hlc}[2][yellow]{{\sethlcolor{#1}\hl{#2}}}
\newcommand{\scripty}[1]{\ensuremath{\mathcalligra{#1}}}

% code for custom comments
\usepackage{tikz}
\usepackage{xcolor}
\usepackage{mathtools}

% markers (black, blue, brown, cyan, darkgray, gray, green, lightgray, lime, magenta, olive, orange, pink, purple, red, teal, violet, white, yellow, more: https://en.wikibooks.org/wiki/LaTeX/Colors)
\newcommand{\marker}[3]{
  \tikz[baseline=(X.base)]{
    \node [fill=#1!40,rounded corners] (X) {#2:};
  }
  {\color{#1!80!black}#3}
}
\newcommand{\khb}[1]{\marker{teal}{KHB}{#1}}  % Kathy Breen
\newcommand{\replyhead}{\textbf{Author Reply: }}

\newfloat{figure}{h}{lot}
\floatname{figure}{Figure}
\newfloat{sfigure}{h}{lot}
\floatname{sfigure}{Figure}
\newfloat{lfigure}{h}{lot}
\floatname{lfigure}{Figure}

\definecolor{shadecolor}{named}{LightGray}

\DeclareMathAlphabet{\mathcalligra}{T1}{calligra}{m}{n}
\DeclareFontShape{T1}{calligra}{m}{n}{<->s*[2.2]callig15}{}

\renewcommand{\thesfigure}{S\arabic{sfigure}}
\renewcommand{\thelfigure}{\Alph{lfigure}}

%-----------------------------------------------------------------------------
% HEADER/FOOTER INFO
%-----------------------------------------------------------------------------
%\pagestyle{fancy}
\fancyhf{}
%\cfoot{\textit{telephone} 217-333-1441 $\bullet$ \textit{fax} 217-333-2736 $\bullet$ \textit{email} matse@illinois.edu}
\cfoot{\thepage}

%----------------------------------------------------------------------------------------
%	SENDER INFORMATION
%----------------------------------------------------------------------------------------

\def\Who{Andrew L. Ferguson} % Your name
\def\What{, PhD} % Your title
\def\Where{Pritzker School of Molecular Engineering\\University of Chicago} % Your department/institution
\def\department{Pritzker School of Molecular Engineering}
\def\institution{University of Chicago}
\def\Address{5640 South Ellis Avenue} % Your address
\def\CityZip{Chicago, IL 60637} % Your city, zip code, country, etc
\def\Email{e:~\url{andrewferguson@uchicago.edu}} % Your email address
\def\TEL{p:~(773)~702-5950} % Your phone number
\def\URL{w:~\url{www.ferglab.com}} % Your URL
\def\titleone{Associate Professor\\ Vice Dean for Equity, Diversity, and Inclusion}

%----------------------------------------------------------------------------------------
%	HEADER AND FROM ADDRESS STRUCTURE
%----------------------------------------------------------------------------------------

\FPmul\tmp{2}{\marginWidth}	% computing institution logo indent based on margin width
\FPsub\logoIndent{7.55}{\tmp}

\address{
\includegraphics[width=3.5in]{UChicago_PME_logo.png} % Include the logo of your institution
%\hspace{\logoIndent in} % Position of the institution logo, increase to move left, decrease to move right
\hspace{2.8in} % Position of the institution logo, increase to move left, decrease to move right
\vskip -0.82in~\\ % Position of the text in relation to the institution logo, increase to move down, decrease to move up
\Large\hspace{0.75in} \hfill ~\\[0.05in] % First line of institution name, adjust hspace if your logo is wide
\hspace{0.75in} \hfill \normalsize % Second line of institution name, adjust hspace if your logo is wide
\makebox[0ex][r]{\bf \Who \What }\hspace{0.65in} % Print your name and title with a little whitespace to the right
~\\[-0.11in] % Reduce the whitespace above the horizontal rule
\hspace{0in}\vhrulefill{1pt} \\ % Horizontal rule, adjust hspace if your logo is wide and \vhrulefill for the thickness of the rule
\hspace{\fill}\parbox[t]{4in}{ % Create a box for your details underneath the horizontal rule on the right
\footnotesize % Use a smaller font size for the details
%\Who \\  % Your name, all text after this will be italicized
%\\
\titleone
%\titletwo\\
%\titlethree\\
%\titlefour\\
\vskip 0.05in
\department\\ % Your department
\institution\\ % Your institution
\Address\\ % Your address
\CityZip
\vskip 0.05in
\TEL\\ % Your phone number
\Email\\ % Your email address
\URL\\ % Your URL
}
\hspace{-1.4in} % Horizontal position of this block, increase to move left, decrease to move right
\vspace{-1in} % Move the letter content up for a more compact look
}

%----------------------------------------------------------------------------------------
%	TO ADDRESS STRUCTURE
%----------------------------------------------------------------------------------------

\FPmul\tmp{2}{\marginWidth}	% computing date indent based on margin width
\FPsub\dateIndent{4.15}{\tmp}

\def\opening#1{\thispagestyle{empty}
{\centering\fromaddress \vspace{1.1in} \\ % Print the header and from address here, add whitespace to move date down
%
\hspace*{4.15 in}\today\hspace*{\fill}\par
} % Print today's date, remove \today to not display it
{\raggedright \toname \\ \toaddress \par} % Print the to name and address
\vspace{0.05in} % White space after the to address
\noindent #1 % Print the opening line
% Uncomment the 4 lines below to print a footnote with custom text
%\def\thefootnote{}
%\def\footnoterule{\hrule}
%\footnotetext{\hspace*{0.18\textwidth}{Telephone 217-333-1441 $\bullet$ Fax 217-333-2736 $\bullet$ email matse@illinois.edu\footnotesize \hspace{3cm} \thepage}}
%\def\thefootnote{\arabic{footnote}}
}

%----------------------------------------------------------------------------------------
%	SIGNATURE STRUCTURE
%----------------------------------------------------------------------------------------

\signature{\Who \What} % The signature is a combination of your name and title

\long\def\closing#1{
\vspace{0.1in} % Some whitespace after the letter content and before the signature
\noindent % Stop paragraph indentation
%\hspace*{\longindentation} % Move the signature right
%\parbox{\indentedwidth}{\raggedright
#1 % Print the signature text
\vskip 0.0in
\includegraphics[width=0.2\textwidth]{signature_720}
%\vskip 0.65in % Whitespace between the signature text and your name
\\
\noindent
\fromsig
\vskip -0.05in
\titleone\\
\department\\
\institution\\
\vskip -0.1in
\noindent cc: M.S.~Jones, B.~Ashwood, A.~Tokmakoff
}

%----------------------------------------------------------------------------------------

\begin{document}

%----------------------------------------------------------------------------------------
%	TO ADDRESS
%----------------------------------------------------------------------------------------

\begin{letter}
{
Prof.\ Hanadi Sleiman \\
Associate Editor, \textit{The Journal of the American Chemical Society}
}

%----------------------------------------------------------------------------------------
%	LETTER CONTENT
%----------------------------------------------------------------------------------------

\opening{Dear Prof.\ Sleiman}

Thank you for your 7-Sep-2021 message regarding our article ``Determining sequence-dependent DNA oligonucleotide hybridization and dehybridization mechanisms using coarse-grained molecular simulation, Markov state models, and infrared spectroscopy'' assigned manuscript ID ja-2021-05219p.R1. We thank you for handling our submission and the anonymous reviewers for their careful reading and critical assessment of our work.

We reproduce below the full text of the reviewer comments in blue along with our responses, reproduction of the changes made to our manuscript, and the locations of these edits. The modifications within our revised submission in response to the reviewer comments are indicated in red text. 

In light of these changes, we hope that our revised submission is suitable for publication in \textit{The Journal of the American Chemical Society}.

\closing{Yours Sincerely}

\end{letter}

%----------------------------------------------------------------------------------------

\clearpage
\newpage

\begin{shaded}
\textbf{Response to Reviewer \# 1}
\end{shaded}

\textcolor{blue}{Recommendation: Publish in JACS without change.
}

\textcolor{blue}{Comments:
Most of my comments were properly addressed. The manuscript, as far as I understand, is suitable for a journal like JACS.
}

\textbf{Author Reply:} We are pleased that the reviewer is satisfied with our modifications and responses and that they are pleased to support publication in JACS. We would like to thank them again for their time, effort, and considered and useful expert comments.



%%%

\begin{shaded}
\textbf{Response to Reviewer \# 3}
\end{shaded}

\textcolor{blue}{Recommendation: Publish in JACS after minor revisions.
}

\textcolor{blue}{Comments: Jones et al. have updated their manuscript in response to reviewers' comments and improved it significantly. I am essentially happy to proceed towards publication.
}

\textbf{Author Reply:} We are pleased that the reviewer finds the manuscript improved and suitable for publication in JACS subject to minor revision.

\textcolor{blue}{
Some remaining issues that I would like to bring to the author's attention:
}

\textcolor{blue}{
(1) The authors have applied the finite size correction of Ouldridge et al. following a comment by reviewer 2, but it hasn't been applied in the correct way. It would be correct to apply it to the melting curves discussed at the bottom of p7/top of p8. Since the sequences are homodimers, the correction to Tm at the simulation concentration would actually be zero; instead, the widths of transitions would change, as noted by Ouldridge et al.
}

\textcolor{blue}{
However, the authors have actually simulated at 3.5x the experimental conc. If the Ouldridge correction - which assumes ideal behaviour - applies, then it should be relatively simple to correct the curves for the concentration difference. Simply divide the phi ratio by 3.5, and then calculate f\_hybridised using the modified phi. If the Tms were previously too low, this change will make them lower still (perhaps by a couple of K), but it is the fairest comparison. It shouldn't affect the rest of the analysis, although it is worth noting that the simulations were always performed "at the melting temperature for a conc of 7mM" - there is no well-defined, concentration-independent melting temperature.
}

\textcolor{blue}{
However, it is not appropriate to apply the Ouldridge correction to data measuring the dissociation time. The authors effectively measure the statistics of dissociation events in infinite dilution - these are unaffected by the factors that lead to the Ouldridge correction. More specifically, the Ouldridge correction accounts for concentration fluctuations neglected in small, canonical simulations. These concentration fluctuations contribute to the hybridisation rates, not the dissociation rates.
}

\textbf{Author Reply:} XXX 

1. Removal of correction. Report updated text here and in main text and SI using roodr code. Maybe something like ``(We note that although the melting temperature of the homodimer is not affected by the Ouldridge et al. finite size corrections [REF], this empirical correction does account for the higher DNA concentration employed in simulation (7 mM) relative to experiment (2 mM). Although we do not do so here, Owczarzy et al. present a prescription to apply concentration corrections to Tm using knowledge of the enthalpy and entropy of duplex formation and helix nucleation [REF; ``Predicting Sequence- Dependent Melting Stability of Short Duplex DNA Oligomers'' Biopoly 44: 217?239, 1997].''. In your response you can also elaborate on your point of using Eqn. 11 in Ouldridge would fix conc effects at same T but not at new Tm so would need new T ramp or knowledge of dH and dS. Close with saying in any case our empirical correction accounts for these effects plus other uncontrolled effects such as model accuracy etc.


2. Adding ``at 7 mM'' at various places in text where necessary in defining Tm.)




\textcolor{blue}{
(2) I realise now that the criterion for dissociation is that both of the central pairs separate to 1.3nm, which is more reasonable than my previous understanding. However, although the authors say that the estimates of kd\^slow are robust to a range of choices for this cutoff, the yellow, red and green curves all have a noticeable shift between 1.3 and 2.0nm (unsurprisingly blue does not, since the GC pairs are in the core). So what exactly does robust mean? Why not just report the data for larger cutoffs?
}

\textbf{Author Reply:} XXX (Must draw some cutoff, our approach was to select physically motivated cutoff and show that results are insensitive to choice. Show that 1.3 nm satisfies these criteria -- results change by some small \% amount in switching cutoff to 2.0 nm so using either value is equally valid.)






\textcolor{blue}{
(3) In their response to point (7) in my original review, the authors talk about "averaging" over all microstates. Do they mean estimating F as
}

\textcolor{blue}{
F = -kT ln sum\_i  exp(-F\_i/kT)
}

\textcolor{blue}{
where i runs over all microstates in the macrostate? This is very different from averaging F\_i, as might be inferred from the text.
}

\textbf{Author Reply:} XXX (Modify SI to describe this procedure as weighted average.)





%%%

\begin{shaded}
\textbf{Response to Reviewer \# 4}
\end{shaded}

\textcolor{blue}{Recommendation: In between publish elsewhere and accept without change.
}

\textcolor{blue}{Comments:
The authors have addressed the comments of all the reviewers very thoughtfully and thoroughly. Although I'm still of the opinion that it's probably just below the level for JACS, I do think it is a very good paper, and wouldn't argue against acceptance if that's the majority view of the reviewers.
}

\textbf{Author Reply:} We thank the reviewer again for their time and effort in reviewing our work, and are pleased that they found all comments to have been thoroughly addressed.

\textcolor{blue}{
PS I'm glad the 7M was just a typo!
}

\textbf{Author Reply:} Us too!




\clearpage
\newpage

\bibliographystyle{achemso}
\bibliography{../references_210514_abv}
\end{document}
