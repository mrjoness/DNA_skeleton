%%%%%%%%%%%%%%%%%%%%%%%%%%%%%%%%%%%%%%%%%
% Professional Formal Letter
% LaTeX Template
% Version 1.0 (28/12/13)
%
% This template has been downloaded from:
% http://www.LaTeXTemplates.com
%
% Original author:
% Brian Moses (http://www.ms.uky.edu/~math/Resources/Templates/LaTeX/)
% with extensive modifications by Vel (vel@latextemplates.com)
%
% License:
% CC BY-NC-SA 3.0 (http://creativecommons.org/licenses/by-nc-sa/3.0/)
%
%%%%%%%%%%%%%%%%%%%%%%%%%%%%%%%%%%%%%%%%%

%----------------------------------------------------------------------------------------
%	PACKAGES AND OTHER DOCUMENT CONFIGURATIONS
%----------------------------------------------------------------------------------------

%\documentclass[11pt,a4paper,draft]{letter} % Specify the font size (10pt, 11pt and 12pt) and paper size (letterpaper, a4paper, etc) 
\documentclass[11pt,a4paper]{letter} % Specify the font size (10pt, 11pt and 12pt) and paper size (letterpaper, a4paper, etc) 
\usepackage{letterbib}
\usepackage{graphicx} % Required for including pictures
\usepackage{microtype} % Improves typography
\usepackage{fancyhdr}
\usepackage[T1]{fontenc} % Required for accented characters
\usepackage[sort&compress,numbers,super]{natbib}
\usepackage{caption}
\usepackage{newfloat}
\usepackage{float}
\usepackage{url}
\usepackage{color}
\usepackage{soul}
\usepackage[svgnames]{xcolor}
\usepackage{framed}
\usepackage{calligra}
\usepackage{fp}
\FPset\marginWidth{0.7}	% margin width (all sides) in inches
\usepackage[margin=\marginWidth in]{geometry}
\usepackage{marginfix}
\usepackage[colorlinks=true,
            linkcolor=blue,
            urlcolor=blue,
            citecolor=blue]{hyperref}
\usepackage[version=3]{mhchem} % Formula subscripts using \ce{}
% Create a new command for the horizontal rule in the document which allows thickness specification
\makeatletter

\def\vhrulefill#1{\leavevmode\leaders\hrule\@height#1\hfill \kern\z@}
\makeatother

%more new commands and definitions
\newcommand*{\rood}[1]{\textcolor{red}{#1}}
\newcommand*{\blauw}[1]{\textcolor{blue}{#1}}
\newcommand*{\groen}[1]{\textcolor{green}{#1}}
\newcommand{\hlc}[2][yellow]{{\sethlcolor{#1}\hl{#2}}}
\newcommand{\scripty}[1]{\ensuremath{\mathcalligra{#1}}}

\newfloat{figure}{h}{lot}
\floatname{figure}{Figure}
\newfloat{sfigure}{h}{lot}
\floatname{sfigure}{Figure}
\newfloat{lfigure}{h}{lot}
\floatname{lfigure}{Figure}

\definecolor{shadecolor}{named}{LightGray}

\DeclareMathAlphabet{\mathcalligra}{T1}{calligra}{m}{n}
\DeclareFontShape{T1}{calligra}{m}{n}{<->s*[2.2]callig15}{}

\renewcommand{\thesfigure}{S\arabic{sfigure}}
\renewcommand{\thelfigure}{\Alph{lfigure}}

%-----------------------------------------------------------------------------
% HEADER/FOOTER INFO
%-----------------------------------------------------------------------------
%\pagestyle{fancy}
\fancyhf{}
%\cfoot{\textit{telephone} 217-333-1441 $\bullet$ \textit{fax} 217-333-2736 $\bullet$ \textit{email} matse@illinois.edu}
\cfoot{\thepage}

%----------------------------------------------------------------------------------------
%	SENDER INFORMATION
%----------------------------------------------------------------------------------------

\def\Who{Andrew L. Ferguson} % Your name
\def\What{, PhD} % Your title
\def\Where{Pritzker School of Molecular Engineering\\University of Chicago} % Your department/institution
\def\department{Pritzker School of Molecular Engineering}
\def\institution{University of Chicago}
\def\Address{5640 South Ellis Avenue} % Your address
\def\CityZip{Chicago, IL 60637} % Your city, zip code, country, etc
\def\Email{e:~\url{andrewferguson@uchicago.edu}} % Your email address
\def\TEL{p:~(773)~702-5950} % Your phone number
\def\URL{w:~\url{http://andrewferguson.uchicago.edu}} % Your URL
\def\titleone{Associate Professor}

%----------------------------------------------------------------------------------------
%	HEADER AND FROM ADDRESS STRUCTURE
%----------------------------------------------------------------------------------------

\FPmul\tmp{2}{\marginWidth}	% computing institution logo indent based on margin width
\FPsub\logoIndent{7.55}{\tmp}

\address{
\includegraphics[width=3.5in]{UChicago_PME_logo.png} % Include the logo of your institution
%\hspace{\logoIndent in} % Position of the institution logo, increase to move left, decrease to move right
\hspace{2.8in} % Position of the institution logo, increase to move left, decrease to move right
\vskip -0.82in~\\ % Position of the text in relation to the institution logo, increase to move down, decrease to move up
\Large\hspace{0.75in} \hfill ~\\[0.05in] % First line of institution name, adjust hspace if your logo is wide
\hspace{0.75in} \hfill \normalsize % Second line of institution name, adjust hspace if your logo is wide
\makebox[0ex][r]{\bf \Who \What }\hspace{0.65in} % Print your name and title with a little whitespace to the right
~\\[-0.11in] % Reduce the whitespace above the horizontal rule
\hspace{0in}\vhrulefill{1pt} \\ % Horizontal rule, adjust hspace if your logo is wide and \vhrulefill for the thickness of the rule
\hspace{\fill}\parbox[t]{4in}{ % Create a box for your details underneath the horizontal rule on the right
\footnotesize % Use a smaller font size for the details
%\Who \\  % Your name, all text after this will be italicized
%\\
\titleone
%\titletwo\\
%\titlethree\\
%\titlefour\\
\vskip 0.05in
\department\\ % Your department
\institution\\ % Your institution
\Address\\ % Your address
\CityZip
\vskip 0.05in
\TEL\\ % Your phone number
\Email\\ % Your email address
\URL\\ % Your URL
}
\hspace{-1.4in} % Horizontal position of this block, increase to move left, decrease to move right
\vspace{-1in} % Move the letter content up for a more compact look
}

%----------------------------------------------------------------------------------------
%	TO ADDRESS STRUCTURE
%----------------------------------------------------------------------------------------

\FPmul\tmp{2}{\marginWidth}	% computing date indent based on margin width
\FPsub\dateIndent{4.15}{\tmp}

\def\opening#1{\thispagestyle{empty}
{\centering\fromaddress \vspace{1.1in} \\ % Print the header and from address here, add whitespace to move date down
%
\hspace*{4.15 in}\today\hspace*{\fill}\par
} % Print today's date, remove \today to not display it
{\raggedright \toname \\ \toaddress \par} % Print the to name and address
\vspace{0.05in} % White space after the to address
\noindent #1 % Print the opening line
% Uncomment the 4 lines below to print a footnote with custom text
%\def\thefootnote{}
%\def\footnoterule{\hrule}
%\footnotetext{\hspace*{0.18\textwidth}{Telephone 217-333-1441 $\bullet$ Fax 217-333-2736 $\bullet$ email matse@illinois.edu\footnotesize \hspace{3cm} \thepage}}
%\def\thefootnote{\arabic{footnote}}
}

%----------------------------------------------------------------------------------------
%	SIGNATURE STRUCTURE
%----------------------------------------------------------------------------------------

\signature{\Who \What} % The signature is a combination of your name and title

\long\def\closing#1{
\vspace{0.1in} % Some whitespace after the letter content and before the signature
\noindent % Stop paragraph indentation
%\hspace*{\longindentation} % Move the signature right
%\parbox{\indentedwidth}{\raggedright
#1 % Print the signature text
\vskip 0.0in
\includegraphics[width=0.25\textwidth]{signature_720}
%\vskip 0.65in % Whitespace between the signature text and your name
\\
\noindent
\fromsig
\vskip -0.05in
\titleone\\
\department\\
\institution\\
\vskip -0.1in
\noindent cc: M.~Jones, B.~Ashwood, A.~Tokmakoff
}

%----------------------------------------------------------------------------------------

\begin{document}

%----------------------------------------------------------------------------------------
%	TO ADDRESS
%----------------------------------------------------------------------------------------

\begin{letter}
{
Prof.\ Erick Carreira \\
Editor-in-Chief, \textit{The Journal of the American Chemical Society}
}

%----------------------------------------------------------------------------------------
%	LETTER CONTENT
%----------------------------------------------------------------------------------------
\newcommand*{\noteb}[1]{\textcolor{blue}{[[#1]]}}		% notes 

\opening{Dear Prof.\ Carreira}

We are pleased to submit a manuscript for consideration for publication in the \textit{The Journal of the American Chemical Society} an original research \textbf{Article} entitled \textbf{``Determining sequence-dependent DNA oligonucleotide hybridization and dehybridization mechanisms using coarse-grained molecular simulation, Markov state models, and infrared spectroscopy''} by Michael S. Jones (U.~Chicago), Brennan Ashwood(U.~Chicago), Andrei Tokmakoff(U.~Chicago), and Andrew L.\ Ferguson (U.~Chicago). This article has not been published previously and is not under consideration in any other journal. All authors have seen and approved the submission of the manuscript.

% background and why it is important to do this.
\textbf{Background.}  A robust understanding of the sequence-dependent thermodynamics of DNA hybridization has enabled rapid advances in DNA nanotechnology. A fundamental understanding of the sequence-dependent kinetics and mechanisms of hybridization and dehybridization remains comparatively underdeveloped. Although previous experimental and computational studies have identified deviations from an "all-or-nothing" hybridization/dehybridization model, few studies have integrated these approaches in order to evaluate and quantify sequence-dependent dynamics.

% what we did in this paper
\textbf{Key Results.}  In this work, we establish new understanding of the sequence-dependent hybridization/dehybridization kinetics and mechanism within a family of self-complementary pairs of 10-mer DNA oligomers by integrating coarse-grained molecular simulation, machine learning of the slow dynamical pathways, data-driven inference of long-time kinetic models, and experimental temperature-jump infrared spectroscopy. We conducted more than 1 ms of unbiased coarse-grained molecular dynamics simulations and employed deep learning techniques to construct high-resolution Markov state models as predictive and interpretable models of the sequence dependent dynamics. Our results reveal that the specific placement of interrupting G:C pairs within an otherwise repetitive AT sequence can have a profound impact on the kinetic pathways and mechanisms for association and dissociation of the DNA duplex.

% impact and importance of this paper
\textbf{Significance.} Our results establish new understanding of the dynamical richness of sequence-dependent kinetics and mechanisms of DNA hybridization/dehybridization, present a molecular basis with which to understand experimental temperature jump data, and furnish foundational design rules by which to rationally engineer the kinetics and pathways of DNA association and dissociation in burgeoning DNA nanotechnology applications.

% fit for the journal
\textbf{Scope.}  This work fits within the scope of \textit{The Journal of the American Chemical Society} in that it reports new fundamental physical chemical understanding of the molecular pathways and mechanisms of DNA hybridization as a function of sequence. These sequence-dependent mechanisms are of increasing importance to both fundamental biophysical and biochemical processes and rapidly developing nanotechnology applications.


% reviewers
\textbf{Reviewers.} We propose the following scientists as well qualified to review this work.

\textbf{\textit{Simulation.}} \noteb{Notes are just to show connection}


Joshua Lequieu \hfill Chemical and Biological Engineering, Drexel University \\
(215) 895-6624 \hfill \url{lequieu@drexel.edu}

$\circ$ \textit{Expert in multi-scale biomolecular simulation with a focus in materials science and biology. \noteb{3spn2}}

Jonathon K. Whitmer \hfill Chemical and Biomolecular Engineering, Notre Dame \\
(574) 631-1417 \hfill \url{jwhitme1@nd.edu}

$\circ$ \textit{Expert in simulations of polyelectrolytes and self-assembling systems.  \noteb{3spn2}}

Thomas Knotts \hfill Chemical Engineering, Brigham Young University \\
(801) 422-9158 \hfill \url{thomas.knotts@byu.edu}

$\circ$ \textit{Expert in theory and simulation of multi-molecular systems.  \noteb{3spn.0 and .1}}

Shoji Takada \hfill Biophysics, Graduate School of Science, Kyoto University \\
+81-(0)75-753-4070 \hfill \url{takada@ biophys.kyoto-u.ac.jp}

$\circ$ \textit{Expert in simulations of DNA-protein systems \noteb{3spn2 and  CG protein modeling}}

Thomas Ouldridge \hfill  Bioengineering, Imperial College London \\
+44-(0)20-7589-5111 \hfill \url{t.ouldridge@implerial.ac.uk}

$\circ$ \textit{Expert in simulations of biochemical systems for nanotechnology applications \noteb{oxDNA}}

Modesto Orozco \hfill  Institute for Research in Biomedicine, University of Barcelona \\
+34-93-40-37156 \hfill \url{modesto.orozco@irbbarcelona.org}

$\circ$ \textit{Expert in molecular dynamics simulations of nucleic acids \noteb{all-atom DNA model}}

\textbf{\textit{Experiment.}}

Anjum Ansari \hfill Physics, University of Illinois at Chicago\\
(312) 996-8735 \hfill \url{ansari@uic.edu}

$\circ$ \textit{Expert in kinetics of biomolecular relaxation from non-equilibrium ensembles.}

Neil Hunt \hfill Chemistry, University of York \\
+44-(0)19-0432-2511 \hfill \url{neil.hunt@york.ac.uk}

$\circ$ \textit{Expert in ultrafast spectroscopy of complex fluids and biological systems.}

Xinsheng Zhao \hfill Chemistry and Molecular Engineering, Peking University \\
+86-10-6275-1727 \hfill \url{zhaoxs@pku.edu.cn}

$\circ$ \textit{Expert in dynamic mechanisms of biomolecular reactions and nucleic acid biophysics.}


We look forward to hearing from you.

\closing{Yours Sincerely}

\end{letter}

%----------------------------------------------------------------------------------------

%\clearpage
%\newpage

%\begin{shaded}
%\textbf{Response to Reviewer \# 1}
%\end{shaded}

%\bibliographystyle{achemso}
%\bibliography{bibliography.bib}

\end{document}
